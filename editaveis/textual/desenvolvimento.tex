\chapter[Desenvolvimento ]{Desenvolvimento}
Esta sessão tratará sobre como foi estabelecido o processo de verificação e
validação de um dado aterfato. Iremos abordar desde a escolha deste artefato até
a sua melhoria com o processos estabelecidos.
\section{Definição do Escopo}
\subsection{Disciplina }

A disciplina escolhida para experimentalmente ser analisada pela equipe é a disciplina de Análise e Design. Essa disciplina tem por finalidade converter a visão unilateral de requisitos em objetos mais palpáveis que trazem uma visão de design do sistema a ser criado. Contudo, trazer também uma visão mais sofisticada da arquitetura para o sistema. [1] 

Nessa disciplina de Análise e Design executam-se 6 microprocessos, onde no processo inicial verifica-se a viabilidade conforme o previsto, e avalia-se as tecnologias disponíveis para auxiliarem a produção. Com isso, o foco é direcionado ao desenvolvimento de uma arquitetura inicial para o sistema, e ai o enfoque passa a ser análise de comportamento e a criação de um conjunto inicial de elementos comportamentais. [2]

\subsection{Produtos da Disciplina}
Com a execução do fluxo de trabalho os seguintes produtos são gerados: 

... [TABELA]

\subsection{Definição do Escopo}
O Modelo de Dados foi definido como o produto gerado da disciplina que usaremos como objeto da verificação. Ele descreve uma representação lógica dos dados persistentes do sistema, alem de trazer em grande parte particularidades do comportamento do banco de dados. [4]

Afim de direcionar a inspeção a seguinte ordem de inspeção e analise é proposta:

\begin{itemize}
\item Entidades
\item Atributos
\item Chaves
\item Relacionamentos
\item Coerência Nominal
\end{itemize}

\section{Definição do Processo de VeV}
\subsection{Escolha da Metodologia Estática}
Para a definição do processo de Verificação e Validação, utilizamos a revisão
estática Walkthrogth. Pois,  a equipe avaliadora é composta por técnicos no escopo
do artefato escolhido. Além disso, com este método estático, a equipe pode corroborar
junta a fim de estabelecer uma melhor verificação para o produto.

\subsection{Processo de VeV}
Imagem*

Descrição*

\subsection{Produtos de Auxílio}
Para desenvolver os trabalhos da revisão estática, são necessários alguns artefatos
de auxílio. Destes, teremos:

\begin{itemize}
\item Metamodelo UML - Documento de embasamento para o estabelecimento da sintaxe
de metamodelo de dados
\item Indagações Individuais - Pontos que cada avaliador do documento julgou como
importante para a melhoria do artefato
\item Checklist - Documento de Revisão, criado para ponderar os pontos acordados
entre as partes para a refatoração do trabalho.
\end{itemize}

\subsection{Infraestrutura Utilizada}
Quanto aos esquipamentos físicos e softwares utilizados para desenvolver a Verificação
e validação do produto, tivemos:

\begin{itemize}
  \item Quatro Notebooks - Ferramentas para desenvolvimento
  \item Software de Modelagem de Dados - Utilizada para representar o modelo de
  banco de dados da aplicação
  \item Software de Modelagem de Processos (Bizagi) - Responsável pelo desenho do
  processo adotado pelo time.
\end{itemize}

\subsection{Participantes}
Para verificar e validar o artefato escolhido, teremos duas equipes. A primeira é
constituída pelo Dono do Produto, que conhece o domínio e que está desenvolvendo
a modelagem dos dados. A segunda trata-se da EVeV, que avaliará o documento
pelo método estático do Walkthrogth.

\subsection{Resultados Esperados}
Após obtermos todo o feedback e refatoração a partir dos pontos levantados no
Documento de Revisão, espera-se uma modelagem de dados mais robusta e que proporcione
maior satisfação para o cliente final.
