\chapter[Desenvolvimento ]{Desenvolvimento}
Esta sessão tratará sobre como foi estabelecido o processo de verificação e
validação de um dado aterfato. Iremos abordar desde a escolha deste artefato até
a sua melhoria com o processos estabelecidos.
\section{Definição do Escopo}

\section{Definição do Processo de VeV}
\subsection{Escolha da Metodologia Estática}
Para a definição do processo de Verificação e Validação, utilizamos a revisão
estática Walkthrogth. Pois,  a equipe avaliadora é compostar por técnicos no escopo
do artefato escolhido. Além disso, com o walkthrough, a equipe pode corroborar
junta a fim de estabelecer uma melhor verificação para o produto.

\subsection{Processo de VeV}
Imagem*

Descrição*

\subsection{Produtos de Auxílio}
Para desenvolver os trabalhos da revisão estática, são necessários alguns artefatos
de auxílio. Destes, teremos:

\begin{itemize}
\item Metamodelo UML - Documento de embasamento para o estabelecimento da sintaxe
de metamodelo de dados
\item Indagações Individuais - Pontos que cada avaliador do documento julgou como
importante para a melhoria do artefato
\item Checklist - Documento de Revisão, criado para ponderar os pontos acordados
entre as partes para a refatoração do trabalho.
\end{itemize}
