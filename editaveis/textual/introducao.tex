\chapter[Roteiro]{Roteiro}

Para executar esta bateria de testes, iremos utilizar algumas ferramentas:

\begin{itemize}
    \item Hping3
    \item Loic
    \item Simple Server UDP
\end{itemize}

Para instalar estas ferramentas, pode ser utilizado um repositório do github,
que contém o script de instalação delas:
\hyperref[label_name]{https://github.com/TiagoAssuncao/test-script-dmz}.

O uso dessas ferramentas serão detalhados logo a seguir, descrevendo os passos para
testar o sistema com estas. Para todas as ferramentas, primeiramente, iremos
iniciar a fase de learning, após, os ataques serão executados.

\section{Hping3 - Fluxo TCP}
\label{sec:Hping3 - Fluxo TCP}
A hping3 será utilizada para testar a aplicação TCP SYN. Os seguintes passos
serão executados:

\begin{enumerate}
    \item Iniciar a fase de learning com o comando seguinte:

        hping3 --syn -c 10000 -p 80 ipDeDestino

    \item Após dez flows, pode-se aplicar o ataque com as seguintes flags:

        hping3 --flood --syn  -p 80 ipDeDestino

    \item Dessa maneira, deve entrar outro atacante com velocidade de ataque menor,
        para que ele apareça nos top senders:

        hping3 --faster --syn  -p 80 ipDeDestino

    \item E por fim, deve-se diminuir a velocidade de ataque do primeiro atacante,
        para que seja possível visualizar a troca de posições no top sender.

        hping3 --syn  -p 80 ipDeDestino
\end{enumerate}

Como resultado destes testes, espera-se que no passo um, o DMZ se comporte sem
nenhum alerta, pois este estará na fase de aprendizado. Logo após, no passo 2,
espera-se um alerta de ataque por parte do programa, sinalizando os ip's que estão
o atacando.

Por fim, os últimos dois passos testarão as funcionalidades que dizem respeito
ao Top Sender.

\section{Loic - HTTP}
\label{sec:Loic - HTTP}
Loic para testar aplicações HTTP. Utilizar o ./loic run.

Este irá iniciar uma interface gráfica, que deve receber os parâmetros de IP
destino, porta, velociade de ataque e protocolo. Este irá verificar todo o
funcionamento avaliado pelo Hping3, porém, no protocolo HTTP. Dessa maneira,
serão repetidos os passos 1 e 2 do teste em TCP Syn.
