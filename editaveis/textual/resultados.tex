\chapter[Resultados Obtidos]{Resultados Obtidos}

\section{Defeitos Encontrados}

\begin{enumerate}
  \item Murilo
  \begin{itemize}
    \item Variável sobrenome deve aceitar mais caracteres
    \item O software vai ser aberto? ou específico para um organização?
    \item Questions? é sobre o projeto? sobre o objetivo? talvez melhorar nome da entidade de relacionamento!
  \end{itemize}
  \item Tiago
  \begin{itemize}
    \item Relacionamentos das tabelas
    \item Nome do atributo calculusDATE
    \item Tamanho das variaveis
    \item Campo text ao inves do varchar
    \item Descrição das métricas
    \item Sintaxe da escrita da tabela  scale
  \end{itemize}
  \item Wilton
  \begin{itemize}
    \item Identificar Relacionamentos
    \item Entidade METRIC possui um atributo confuso
    \item Qual a necessidade da entidade Scale?

  \end{itemize}
\end{enumerate}

\section{Criticidade dos defeitos encontrados}

O nível de criticidade dos defeitos apontados pela equipe de verificação e validação pode ser
considerado baixo. Já que o modelo de dados analisado encontrava-se em um nível bem elevado de maturidade.
Desta forma as correções que foram apontadas foram apenas há nível de refatoramento de Entidades e atributos.
Descrição suscinta de relacionamentos entre enticedades.

\section{Necessidade de reparos e Ações corretivas}

A necessidade de execução dos reparos levantados pela equipe de verificação e validação se faz necessáro para evitar
a ocorrência de erros futuros no projeto devido ao não entendimento ou entendimento incompleto do modelo de dados.
Dado que quanto mais tempo um erro leva para ser descoberto em um projeto de software, maior será o custo para reparar o mesmo.
Faz se extremamente necessário o reparo dos erros neste momento.

\section{Responsáveis e prazos de correção}

Baseado no processo definido, o responsável pelas correções levantadas pela equipe de verificação e validação será a integrante Karine
Valença. No período de 48 horas com base no documento de revisão gerado pelos demais integrantes.


\section{Dicotomia no posicionamento da Explanação}

A metodologia estática Walkthrough não estabelece um processo para a sua execução, dessa forma, os passos seguidos são escolhidos pela equipe que irá executar a inspenção. A maioria das equipes escolhe fazer a apresentação do produto antes da leitura individual e do levantamento das indagações individuais.

A nossa equipe tomou como base o conceito que uma explicação é muito mais proveitosa quando a equipe que está auditando tem conhecimento prévio do trabalho.
Dessa maneira, tomamos a decisão de iniciar uma análise individual do artefato
antes da Explanação.

Acreditamos que dessa maneira, conseguimos ter um proveito maior no momento da
explicação por parte do dono do produto.
